\section{The Story}

MadNetwork has progressed through many iterations to bring the project
to where we are today.
In each of those iterations, we learned what could work and arguably,
more importantly, what would never work.
In an early pilot of the technology we leverage the Ethereum blockchain
as a means to rapidly iterate design patterns.
Although the Ethereum network is an outstanding piece of technology
that will forever change the world, we kept finding the sharp edges
while using the technology in the enterprise setting.
The costs of storage, the inherent limitations on throughput, the
complications around cleaning up stale state, and the complexity of
integrating enterprise partners into the technology all caused
complications.
These efforts led us to research other systems that addressed the
concerns we were experiencing.
What we found was that these problems are by no means isolated.
In fact, what we found was that the problems were fairly universal.

This early pilot project was a primitive form of our end goal.
That goal was to build a better public key infrastructure (PKI).
Given an application specific PKI, all members of the advertising
supply chain can be assigned strongly unforgeable identities.
By not linking the resolution of these keys to a presupposition that
all members of the AdTech supply chain must have a registered internet
domain, we open the possibility of using this system to more
technologies.
These identities may be used for many purposes, but chief among them is
proofs of integrity and authenticity for messages communicated through
third parties.
In this way, we may prevent many forms of fraud and gain insight into
the supply chains themselves.
These systems also allow the negotiation and/or distribution of
encryption keys between parties.
In the context of advertising this means we can build broadcast
encryption technologies that allow private data to be streamed through
openRTB such that only the intended party/parties may decrypt it.
This capability is sorely needed due to the recent industry
transformations being forced by GDPR, CCPA, and other privacy-centric
legislation.
Additional capabilities are also possible due to the inclusion of
pairing-based cryptography.
These include Hierarchical Identity Based Cryptography for the use of
securely sharing identities between partners such that external groups
are not needed for matching operations.
In summary, this system also allows network participants to communicate
private data in the oRTB environment, as well as communicate identity
of the user in the bid request to opted-in recipients.

Although x.509 stands as the de facto standard for enabling enterprise
encryption and server authentication, this system is built on
antiquated technologies that have been the root of many problems in the
past two decades.
Further, x.509 is fundamentally ill-equipped to perform all but the
most basic of what was needed to fully address the problems of the
AdTech industry.
Thus, a new system is needed to combat the concerns of AdTech.
This is once again why the IAB has begun work on ads.cert.
Asymmetric cryptography would appear to be the best solution we have to
protecting digital commerce.
This should come as no surprise.
What should be surprising is that a high tech industry such as AdTech
has not managed to implement a solution already.

In order to build such a PKI using the Ethereum blockchain, smart
contracts were leveraged that allowed a root of trust to associate
cryptographic identities with real world businesses.
This decentralized design meant that the cryptographic identity of any
registered business could be referenced in a fully transparent system,
without the need to contact our systems in the process.
The other benefit to such a design was that unlike x.509 where
revocation and transparency have been a pain point for years, we
inherited these properties by default from day one.
This system also afforded the ability for any registered entity to
register new cryptographic keys, for many different applications, with
no interaction from our systems outside initial registration.
This system even allowed those safe keys to be revoked by simply
removing them from the blockchain.

What we were building was analogous to modern technologies intended to
shore up x.509 against rampant fraud that occurred in the last decade.
Specifically, technologies such as Certificate Transparency.
Unfortunately, simply using the core concepts of Certificate
Transparency still did not address the issue of revocation.
Considering the vast fraud that already plagues the AdTech industry,
any PKI solution that did not accommodate efficient revocation and
scale to millions of requests per second was a liability, not a
solution.
This inability to scale to millions of requests is the very failure of
OCSP, another x.509 technology that was intended to address revocation.
Fortunately, there does exist a mechanism to address the problem of
revocation.
This solution will be covered during the formal discussion of the PKI
we have built.

Although the Ethereum system did work for the purposes of a pilot
project, bringing thousands of new enterprise systems into the fold
would eventually prevent any solution built on this technology from
succeeding due to the scaling constraints of the underlying system.
What we needed was a system that was built for the specific purpose of
creating a better PKI.
What we needed was a cryptographically verifiable map that could expire
stale records automatically, was capable of sharding so that it could
grow smoothly with system utilization, was inexpensive to write data
into, and had mechanisms through which tokens could be abstracted for
enterprise partners.
What we needed was MadNetwork.

Unfortunately, such a system did not exist at that time.
The lack of such a system seems astonishing given the trends in
enterprise adoption of blockchain technology.
Many enterprise projects do not require the full complexity of a smart
contract enabled blockchain.
Further, many enterprise projects require that data not be forever
immutable, such as that mechanism that is afforded by Bitcoin's
OpReturn method of writing data.

In order to accommodate the above challenges, while also acknowledging
that creating strong consensus algorithms is a nontrivial task,
MadNetwork has been built as an Ethereum sidechain utilizing a Proof of
Stake consensus algorithm.
By anchoring MadNetwork into an Ethereum smart contract system, those
same properties that otherwise make Ethereum a difficult technology in
the enterprise setting may be leveraged for the benefit of creating a
more secure system.
These benefits include the ability to codify complex governance and
fail-safe mechanisms that would otherwise be incredibly difficult in
the stand-alone Proof of Stake setting.

MadNetwork has been designed to facilitate building a fully-auditable
PKI that allows for efficient revocation as well as compact proofs of
certificate non-revocation.
This system has also been built to allow proofs of the state of the
sidechain to be verified within the Ethereum virtual machine.
Our hope is that, outside of the industry partners we are already
building solutions for, other members of the Ethereum ecosystem may
also leverage this technology to support their own projects.
If this were to occur, it would allow what is the effective equivalent
of smart contract state cold storage to be leveraged within the
Ethereum blockchain.
This would ultimately help address some of the very scaling issues we
experienced while using Ethereum in the past.
