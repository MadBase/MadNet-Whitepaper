\subsection{Distributed Key Generation Protocol}
\label{ssec:dkg}

We desire a Byzantine Fault Tolerant consensus algorithm.
So, we let $\mathcal{P}$ be the total collection of participants with
$\abs{\mathcal{P}} = n$.
We set the threshold $t = \ceil{2n/3} - 1$ in our $\parens{t,n}$
secret sharing protocol.
Thus, it takes $t+1$ users to reconstruct a secret, which corresponds
to strictly greater than two-thirds of the participants.
We assume there is an open broadcast channel between all participants.
Encryption will be provided through Diffie-Hellman style shared
secret encryption; this will be discussed in Sec.~\ref{ssec:secret_enc}.
The group shared secret, henceforth called the \emph{master secret key},
will be the sum of the shared secrets of
each group member who correctly shared his secret.
Once there are $t+1$ valid partial signatures, these will be
combined to form a group signature.

As stated above, although the final master public key will
reside in $\G_{2}$, because the precompiled contracts
currently available in the Ethereum Virtual Machine only allow addition
and scalar multiplication in $\G_{1}$ (multiplication and exponentiation
in our multiplicative notation), we will primarily
use computations in $\G_{1}$ and anything required
in $\G_{2}$ will be confirmed via a \textsc{PairingCheck} call.

\subsubsection{Participant Setup}
Each participant $P_{i}\in\mathcal{P}$ begins by selecting
a secret key $\sk_{i}\in\Z_{q}$ with public key $\pk_{i} = g_{1}^{\sk_{i}}$.
The public-private key pair $\angles{\pk_{i},\sk_{i}}$ will be
used for secure communication over the insecure broadcast channel;
it will not be used for signing any messages.

\subsubsection{Verifiable Secret Sharing}

Participant $P_{i}$ chooses a secret
$s_{i}\in\Z_{q}$ to share with the other participants.
To do this, choose a secret polynomial $f_{i}:\Z_{q}\to\Z_{q}$
with

\begin{equation}
    f_{i}(x) = c_{i0} + c_{i1}x + c_{i2}x^{2} + \cdots + c_{it}x^{t},
\end{equation}

\noindent
where $c_{i0} = s_{i}, c_{i1}, \cdots, c_{it}$ are chosen uniformly
in $\Z_{q}$.
Setting

\begin{equation}
    C_{ik} = g_{1}^{c_{ik}}\quad k\in\braces{0,\cdots,t},
\end{equation}

\noindent
we have the corresponding public polynomial $F_{i}:\Z_{q}\to\G_{1}$:

\begin{equation}
    F_{i}(x) = C_{i0}C_{i1}^{x}\cdots C_{it}^{x^{t}}.
\end{equation}

\noindent
The shared secret from $P_{i}$ to $P_{j}$ is $s_{i\to j} = f_{i}(j)$
and

\begin{equation}
    \overline{\texttt{s}}_{i\to j} =
        \textsc{Encrypt}(\text{sk}_{i},\text{pk}_{j},j, s_{i\to j})
\end{equation}

\noindent
refers to a particular encryption scheme we discuss
in Sec.~\ref{ssec:secret_enc}.
Participant $P_{i}$ will broadcast the message

\begin{equation}
    \left\{ 
        \overline{\texttt{s}}_{i\to 1}, \overline{\texttt{s}}_{i\to 2},
            \cdots,
            \overline{\texttt{s}}_{i\to i-1},
            \overline{\texttt{s}}_{i\to i+1},
            \cdots,
            \overline{\texttt{s}}_{i\to n},
        C_{i0}, C_{i1}, \cdots, C_{it}
    \right\}
\end{equation}

\noindent
over the broadcast channel.
We note this message does not include the secret
$\overline{\texttt{s}}_{i\to i}$.

Once participant $P_{j}$ receives the message from $P_{i}$,
he sets

\begin{equation}
    \hat{s}_{i\to j} = \textsc{Decrypt}(\text{sk}_{j},\text{pk}_{i}, j,
        \overline{\texttt{s}}_{i\to j}).
\end{equation}

\noindent
$P_{j}$ then determines if

\begin{equation}
    g_{1}^{\hat{s}_{i\to j}} \overset{?}{=} F_{i}(j).
    \label{eq:secret_share_test}
\end{equation}

\noindent
If we have equality, then $\hat{s}_{i\to j} = s_{i\to j}$.
Otherwise, $P_{i}$ incorrectly shared his secret.



\subsubsection{Malicious shares}

We now suppose that $\overline{\texttt{s}}_{i\to j}$ is incorrect;
that is, we do not have equality in Eq.~\eqref{eq:secret_share_test}.
In order to prove this to be the case, everyone needs to be
able to prove that the encrypted secret
$\overline{\texttt{s}}_{i\to j}$ is incorrect.
To do this, $P_{j}$ must publish and prove the shared secret $k_{ij}$;
this is required in order to ensure bad actors
do not submit false proofs against honest actors.

Proving $k_{ij}$ is the shared secret is based on showing

\begin{equation}
    \pk_{j} = g_{1}^{\sk_{j}} \quad\text{and}\quad
    k_{ij} = \pk_{i}^{\sk_{j}}
\end{equation}

\noindent
\emph{without} sharing the secret key $\sk_{j}$;
that is, we wish to show
$\dlog_{g_{1}}(\pk_{j}) = \dlog_{\pk_{i}}(k_{ij})$
while keeping their common value ($P_{j}$'s secret key $\sk_{j}$) secret.
To do this, we use a zero-knowledge proof;
see Alg.~\ref{alg:zk_dleq_proof} for constructing the zk-proof
and Alg.~\ref{alg:zk_dleq_verify} for proof verification.
One reference for zk-proofs involving discrete logarithms
is~\cite{camenisch1997proof}.

Thus, $P_{j}$ would compute

\begin{equation}
    \pi(k_{ij}') = \textsc{DLEQ}(g_{1},\pk_{j},\pk_{i},k_{ij}',\sk_{j})
\end{equation}

\noindent
and publish $\angles{k_{ij}',\pi(k_{ij}')}$, where $k_{ij}'$
is claimed shared secret.
This allows anyone to use \textsc{DLEQ-verify} to determine
its validity.
If

\begin{equation}
    \textsc{DLEQ-verify}(g_{1},\pk_{j},\pk_{i},k_{ij}',\pi(k_{ij}'))
        = \texttt{true},
\end{equation}

\noindent
then $k_{ij}' = k_{ij}$, the shared secret.
Using this, everyone can decrypt $\overline{\texttt{s}}_{i\to j}$
by

\begin{equation}
    \hat{s}_{i\to j}
        = \textsc{DecryptSS}(k_{ij},j,\overline{\texttt{s}}_{i\to j}, b)
\end{equation}

\noindent
and determine if

\begin{equation}
    g_{1}^{\hat{s}_{i\to j}} \overset{?}{=} F_{i}(j).
    \label{eq:secret_share_test_2}
\end{equation}

\noindent
If the DLEQ proof $\pi(k_{ij}')$ shows $k_{ij}'$ is the shared
secret between $P_{j}$ and $P_{i}$ and we do not have equality in
Eq.~\eqref{eq:secret_share_test_2}, then $P_{i}$ is acted maliciously
and should be removed.
There are two other possibilities:
$k_{ij}'$ is not the shared secret, or $k_{ij}'$ is the shared secret
and we have equality in Eq.~\eqref{eq:secret_share_test_2}.
In both cases, $P_{j}$ acted maliciously and should be removed.
Thus, when $P_{j}$ submits a claim that $P_{i}$
failed to share a secret,
either $P_{j}$'s or $P_{i}$'s stake will be slashed.

In practice, $P_{j}$ will submit $P_{i}$'s broadcast message to an
Ethereum smart contract along with purported shared secret $k_{ij}'$
and proof $\pi(k_{ij}')$, and the smart contract would
determine its validity and burn stake as appropriate.

\begin{algorithm}[p]
    \caption{Zero-knowledge proof of discrete-logarithm equality.}
\label{alg:zk_dleq_proof}
\begin{algorithmic}[1]
\Function{DLEQ}{$x_{1}$,$y_{1}$,$x_{2}$,$y_{2}$,$\alpha$}
    \Comment{Construct zk-proof that $y_{1} = x_{1}^{\alpha}$
        and $y_{2} = x_{2}^{\alpha}$.}
    \State $w\in_{R}\Z_{q}$
    \Comment{$x_{i}, y_{i}\in\G_{1}$ with $\abs{\G_{1}}=q$.}
    \State $t_{1} = x_{1}^{w}$
    \State $t_{2} = x_{2}^{w}$
    \State $\texttt{x1} = \texttt{u2b}(x_{1})$
    \State $\texttt{y1} = \texttt{u2b}(y_{1})$
    \State $\texttt{x2} = \texttt{u2b}(x_{2})$
    \State $\texttt{y2} = \texttt{u2b}(y_{2})$
    \State $\texttt{t1} = \texttt{u2b}(t_{1})$
    \State $\texttt{t2} = \texttt{u2b}(t_{2})$
    \State $\texttt{c}  = H(\texttt{x1}||\texttt{y1}||\texttt{x2}||\texttt{y2}||\texttt{t1}||\texttt{t2})$
    \State $c = \texttt{b2u}(\texttt{c})$
    \State $r = w - \alpha c \mod q$
    \State $\pi = \parens{c,r}$
    \State \Return $\pi$
\EndFunction
\end{algorithmic}
\end{algorithm}

\begin{algorithm}[p]
    \caption{Zero-knowledge verification of discrete-logarithm equality proof.}
\label{alg:zk_dleq_verify}
\begin{algorithmic}[1]
\Function{DLEQ-verify}{$x_{1}$,$y_{1}$,$x_{2}$,$y_{2}$,$\pi$}
    \Comment{Determine validity of proof from \texttt{DLEQ}}
    \State $\parens{c,r} = \pi$
    \State $t_{1}' = x_{1}^{r}y_{1}^{c}$
    \State $t_{2}' = x_{2}^{r}y_{2}^{c}$
    \State $\texttt{x1}  = \texttt{u2b}(x_{1})$
    \State $\texttt{y1}  = \texttt{u2b}(y_{1})$
    \State $\texttt{x2}  = \texttt{u2b}(x_{2})$
    \State $\texttt{y2}  = \texttt{u2b}(y_{2})$
    \State $\texttt{t1p} = \texttt{u2b}(t_{1}')$
    \State $\texttt{t2p} = \texttt{u2b}(t_{2}')$
    \State $\texttt{cp}  = H(\texttt{x1}||\texttt{y1}||\texttt{x2}||\texttt{y2}||\texttt{t1p}||\texttt{t2p})$
    \State $c' = \texttt{b2u}(\texttt{cp})$
    \If {$c = c'$}
        \State \Return \texttt{true}
    \Else
        \State \Return \texttt{false}
    \EndIf
\EndFunction

\end{algorithmic}
\end{algorithm}


