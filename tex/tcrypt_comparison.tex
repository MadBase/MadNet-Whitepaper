\subsection{Comparision with Ethereum Distributed Key Generation Paper}
\label{ssec:ethdkg_comparison}

In~\cite{ethdkg}, the authors presented the Ethereum
Distributed Key Generation whitepaper.
Here, participants work together to form a master public key
(the public key for the group)
based on verifiable secret sharing.
The master secret key (the private key for the group)
is the summation of the shared secrets correctly shared by
valiadators.
This system may proceed even with a limited number of Byzantine actors.
In the end, participants may compute partial signatures of a message
which may be combined to form a valid group signature.

While we follow the procedure of~\cite{ethdkg} in general,
there are other concerns that must be taken into consideration.
The MadNetwork will be a sidechain of Ethereum,
and the keys we construct will be used to sign the blocks of our sidechain.
By anchoring into Ethereum, we are able to use smart contracts to enforce
compliance with the consensus algorithm as well as punish
those who behave in a cryptographically-verifiable malicious way.
Malicious behavior include submitting false keys, submitting
false proofs, signing invalid messages, and similar actions.
This is different from the original paper where everything
happened on Ethereum and participants who
acted maliciously during the key derivation stage would still
be allowed to proceed because honest actors would be able to
work together to derive the correct information.
In that setting, the focus was having a $\parens{t,n}$-thresholded
system whereby $t+1$ actors are required to work together to
sign messages for the group.
Here, $t$ and $n$ are preassigned.
In our case, we specifically desire a Byzantine-fault tolerant
thresholded system, whereby we require $t = \ceil{2n/3}-1$.
Even though we have the same $t$ values for $n = 3k$ and $n = 3k-1$
(thereby potentially allowing one malicious validator to be
ejected without forcing a required restart),
we will restart the DKG process whenever malicious behavior
is cryptographically proven.
By forcing a restart upon pernicious action,
validators are discouraged from malicious activity as well
as lose the opportunity to earn block rewards.
We also restart when validators fail to submit the required
information.
In this case when malicious intent cannot be proven due to
the possiblity of technical failure, a minor fine will be
given due to time cost of the other participants.

During the DKG process, all of the material necessary
to recover a participant's secret is available provided
enough actors work together.
As mentioned in~\cite{ethdkg}, this is useful in order to allow
for the DKG to continue even if certain participant's fail to cooperate,
but the problem is that with this secret information it would
be possible for a large enough malicious subset (specifically,
a majority greater than two-thirds) to recover a secret
and produce valid signatures from participant $P_{i}$
proving malicious behavior that is not, in fact, perpetrated by $P_{i}$.
This is of serious concern because participants
stake tokens on the basis of being validators
on our blockchain and receiving block rewards for their computation
and are threatened with losing stake should they
behave nefariously.

Requiring at least 67\% percent of the validators to work together
in order to produce stake-burning results is
better than a 51\% Attack which can occur
in Proof-of-Work blockchains but it leaves something to be desired;
we would like for honest validators to be immune to
the previously-mentioned behavior even if there is only
one honest participant.
To combat this, we will require all messages to be signed
by a participant's Ethereum's private key.
This will allow all messages to be validated against the Ethereum
public key and will be safe so long as the Ethereum private key
is secure.
In this way, any secret information leaked during the distributed
key generation will not enable Byzantine actors to produce
cryptographic proof of malicious behavior against honest participants.

