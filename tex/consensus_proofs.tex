\subsubsection{Proof of Safety}

The proof of safety relies on this observation from
Lemma~\ref{lem:threshold_subsets}: two subsets with at least
a threshold number of validators in each subset will share at least one
honest validator.

\begin{lem}
\label{lem:threshold_subsets}
Let $f$ be the number of malicious validators.
Given any two subsets with at least $2f+1$ validators in a system
containing $3f+1$ validators, those two subsets share at least $f+1$ validators.
Thus, these two subsets share at least one honest validator.
\end{lem}

\begin{proof}
Let $A$ and $B$ be two subsets with at least $2f+1$ validators
chosen from $N = 3f+1$ validators.
From De Morgan’s laws, we know

\begin{equation*}
    A \cap B = (A^{c} \cup B^{c})^{c}.
\end{equation*}

\noindent
Here, $A^{c}$ denotes the complement of $A$ (that is, the elements of the
system not in A).
In what follows, $\abs{A}$ denotes the number of elements in $A$.
Thus, we have

\begin{align*}
    \abs{A\cap B} &= \abs{(A^{c} \cup B^{c})^{c}} \\
        &= N - \abs{A^{c} \cup B^{c}} \\
        &\ge N - \parens{\abs{A^{c}} + \abs{B^{c}}} \\
        &\ge N - 2f \\
        &= f + 1.
\end{align*}

\noindent
Their intersection shares at least $f+1$ validators and there are $f$
malicious validators, so we see that $A$ and $B$ share at least one honest
validator.
\end{proof}

This lemma ensures that honest participants will always agree when
PreCommitting a proposal or submitting a NextHeight message in a given
round, as the next two lemmas show.

\begin{lem}
\label{lem:2_honest_precommit}
In any round, any 2 honest participants who PreCommit a proposal will PreCommit
the same proposal.
\end{lem}

\begin{proof}
Suppose two honest participants PreCommit proposals $P_{1}$ and $P_{2}$.
To PreCommit a proposal, they both must have knowledge of at least $2f+1$
PreVote messages from participants.
There corresponds subsets $S_{1}$ and $S_{2}$, where $S_{1}$ contains
the participants who submitted prevotes for $P_{1}$ and $S_{2}$ contains
the participants who submitted PreVotes for $P_{2}$.
By Lemma~\ref{lem:threshold_subsets},
$S_{1}$ and $S_{2}$ share at least one honest participant.
An honest participant will only PreVote for one value in a round, so the
proposals $P_{1}$ and $P_{2}$ must agree.
\end{proof}

\begin{lem}
\label{lem:2_honest_nextheight}
In any round, any 2 honest participants who submit a NextHeight message
will submit a NextHeight message for the same proposal.
\end{lem}

\begin{proof}
Mutatis mutandis, the proof is the same as that of
Lemma~\ref{lem:2_honest_precommit}.
\end{proof}

We are now able to show the multiple distinct NextHeight messages from honest
participants are not able to occur before the DeadBlockRound.
This is a stronger result than Lemma~\ref{lem:2_honest_nextheight}
and follows from the fact that a
threshold number of LockedValues must occur before submitting a NextHeight
message.

\begin{lem}
\label{lem:2_distinct_nextheight}
In any round before the DeadBlockRound, any 2 honest participants
who submit a NextHeight message will submit a NextHeight message
for the same proposal.
\end{lem}

\begin{proof}
Suppose we have not yet reached the DeadBlockRound.
Let $P_{1}$ and $P_{2}$ be NextHeight messages from two honest participants
with corresponding signing subsets $S_{1}$ and $S_{2}$.
When $P_{1}$ is signed, all members of $S_{1}$ are supposed to set
LockedValue to the proposal corresponding to $P_{1}$ because they
PreCommitted $P_{1}$.
Similarly, all members of $S_{2}$ are supposed to set LockedValue
to the proposal corresponding to $P_{2}$ because they PreCommitted $P_{2}$.
This implies $S_{1}$ and $S_{2}$ are two subsets of participants of size
at least $2f+1$ with LockedValue set.
Once an honest participant sets his LockedValue, it is never unset before the
DeadBlockRound.
By Lemma~\ref{lem:threshold_subsets},
$S_{1}$ and $S_{2}$ have at least one honest participant in common
with his LockedValue set to one proposal.
Thus, we see that the $P_{1}$ and $P_{2}$ NextHeight messages must be
for the same proposal when they are submitted by honest participants.
\end{proof}

Lemma~\ref{lem:honest_signed_nextround}
is useful in bounding the number of honest validators who may not
proceed to the next round.

\begin{lem}
\label{lem:honest_signed_nextround}
In order for $2f+1$ validators to enter a round, at least $f+1$
honest validators must have signed a NextRound message and at most $f$
honest validators failed to sign a NextRound message.
\end{lem}

\begin{proof}
We recall an honest validator may not enter a round without a valid
RoundCertificate.
A NextRound message contains two objects.
The first object is a RoundCertificate for the current round.
The second is a RoundShare object for the next round.
A RoundCertificate contains both round number and block height as internal
fields, and in order to form a valid Round Certificate, at least $2f+1$
validators must have signed a RoundShare.
Because there are at most $f$ dishonest validators who signed for the
RoundCertificate, at least $f+1$ honest validators must have also signed the
RoundCertificate.
Thus, at most $f$ honest validators did not sign the RoundCertificate.
\end{proof}

Once validators reach the DeadBlockRound, they will not acknowledge any
NextHeight messages for any preceding round.
This ensures that dishonest validators are not able to fork the chain by
producing a block based on those NextHeight messages.

\begin{lem}
\label{lem:sign_rc_dbr}
If at least $2f+1$ participants sign the RoundCertificate to the DeadBlockRound,
then it is not possible for $2f+1$ participants to form a valid NextHeight
message for any round preceding the DeadBlockRound.
\end{lem}

\begin{proof}
Given Lemma~\ref{lem:honest_signed_nextround}
and the rule that any honest validator who has signed a
NextRound message for the DeadBlockRound will never sign or acknowledge any
NextHeight message from a previous round, at least $f+1$ honest validators must
have signed a NextRound message for the DeadBlockRound if there exists a
RoundCertificate for the DeadBlockRound.
If at least $f+1$ honest validators are in the DeadBlockRound,
at most $f$ honest validators may remain in a round preceding
the DeadBlockRound.
If at most $f$ honest validators remain in a preceding round, then the malicious
validators are unable to use the signatures of $f$ honest validators to form a
set of $2f+1$ valid NextHeight messages.
Therefore, the malicious validators may not form a block from those NextHeight
messages.
\end{proof}

With this assurance, we allow for participants to safely proceed to the
DeadBlockRound even if they previously were locked onto a NextHeight message.

\begin{lem}
\label{lem:safe_unlock_nh_dbr}
It is safe for any participant who is locked on a NextHeight message to unlock
and proceed to the DeadBlockRound upon receiving a RoundCertificate for the
DeadBlockRound.
\end{lem}

\begin{proof}
This follows from Lemma~\ref{lem:sign_rc_dbr}.
\end{proof}

The previous work allows us to show that we will converge to the EmptyBlock
when a RoundCertificate for the DeadBlockRound exists.
This ensures new blocks will be created even if no transactions are performed.

\begin{lem}
\label{lem:emptyblock_rc_dbr}
The DeadBlockRound must converge to the EmptyBlock if a RoundCertificate for
the DeadBlockRound exists.
\end{lem}

\begin{proof}
Upon entering the DeadBlockRound, every valid process will immediately PreVote
the EmptyBlock and ignore all other contradicting votes.
These contradicting votes include any PreVote for a Proposal that is not the
EmptyBlock, any NextRound message, any PreCommit that is not a PreCommit for
the EmptyBlock, and any PreVoteNil or PreCommitNil message as well.
As a result of Lemma~\ref{lem:sign_rc_dbr},
at least $f+1$ honest validators enter the DeadBlockRound.
Therefore, the honest validators who enter the DeadBlockRound will only
progress once at least $f$ other validators also PreVote in the DeadBlockRound.
Because we assume all messages are eventually received, the at most $f$ honest
validators who did not sign the NextRound Certificate for the DeadBlockRound
will eventually receive the RoundCertificate for the DeadBlockRound and PreVote
for the EmptyBlock in the DeadBlockRound.
It follows that the round eventually converges to the EmptyBlock and no other
possible block.
\end{proof}

The previous work also allows us to show we will not converge to more
than one valid block at a given block height.

\begin{lem}
It is not possible for our system to converge to more than one valid block for
any given block height.
\end{lem}

\begin{proof}
If the round enters the DeadBlockRound, then Lemma~\ref{lem:emptyblock_rc_dbr}
shows that we will converge to the EmtpyBlock.
Lemma~\ref{lem:2_distinct_nextheight}
proves that it is not possible to have more than one proposal for which
an associated NextHeight message has been validly formed before the
DeadBlockRound.
Lemma~\ref{lem:safe_unlock_nh_dbr}
allows for a safe transition into the DeadBlockRound.
\end{proof}

The next few lemmas assure the behavior of honest validators as it relates to
voting.

\begin{lem}
\label{lem:honest_must_prevote}
An honest validator who enters a round must eventually PreVote or PreVoteNil in
that round.
\end{lem}

\begin{proof}
All honest validators will either PreVote or PreVoteNil at the termination of
the ProposalTimeout.
Therefore, they must eventually prevote.
\end{proof}

\begin{lem}
\label{lem:2fp1_precommit}
As long as $2f+1$ honest validators enter a round, there will be at least
$2f+1$ PreCommits or PreCommitNils in that round.
\end{lem}

\begin{proof}
By Lemma~\ref{lem:honest_must_prevote},
all honest validators will eventually PreVote or PreVoteNil in a round.
In the event that a PreVote is received for a competing Proposal from the
perspective of a validator who has already PreVoted, that validator will count
this PreVote as a PreVoteNil.
From there, the honest validators will be able to either PreCommit or
PreCommitNil, thus leading to $2f+1$ PreCommits or PreCommitNils.
\end{proof}

\begin{lem}
\label{lem:2fp1_nextround}
As long as $2f+1$ honest validators enter a round, there will be at least
$2f+1$ NextRound or NextHeight messages in that round.
\end{lem}

\begin{proof}
Mutatis mutandis, the proof is the same as that of
Lemma~\ref{lem:2fp1_precommit}.
\end{proof}

We now show that a round must terminate or a higher block is formed.

\begin{lem}
\label{lem:valid_rouncert_exists}
If at any time a valid RoundCertificate exists for a round, that round must
eventually terminate or a higher block must be formed.
\end{lem}

\begin{proof}
We first focus on the case when no validator has signed a DeadBlockRound
RoundCertificate.
In this case, we may have that $f+1$ honest validators have signed a NextHeight
message and the round cannot terminate but a new block will be formed because
the validators will eventually observe the previous NextHeight messages and
will follow them.
Otherwise, at least $f+1$ honest validators will have entered the current round
and it must eventually terminate; we will fall back to the previous case if any
of these validators observe a NextHeight message.

In the case that a DeadBlockRound RoundCertificate exists, we have already
proven termination by Lemma~\ref{lem:emptyblock_rc_dbr}.
\end{proof}

Taken together, we are now able to show that our blockchain will make forward
progress provided there are a limited number of faults.

\begin{lem}
Our blockchain will always make forward progress so long as there are no more
than $f$ faults in the system.
\end{lem}

\begin{proof}
If there are at most $f$ faults, then all rounds must terminate or a new block
will be formed by Lemma~\ref{lem:valid_rouncert_exists}.
\end{proof}
