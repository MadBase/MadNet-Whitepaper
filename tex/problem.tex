\section{The Problem}
MadNetwork was conceptualized as a mechanism to address many of the
long-standing issues surrounding the modern advertising industry.
The number of problems that plague this space are no secret.
For instance, it has been estimated that in 2017 nearly 40\%  of all
programmatic impressions were fraudulent~\cite{DigitalAdFraud2018}.
In 2018 that same fraud accounted for more than \$19 Billion USD
worldwide~\cite{DigitalAdFraud2018}.
These numbers seem unimaginably large already, but fraud is actually
increasing at an exponential rate~\cite{DigitalAdFraud2018}.
The problem will get much worse before it gets better.

Although many solutions have been proposed to address this fraud, they
have proven unable to make a meaningful impact as of now.
A prime example of this lack of innovation rests in ads.txt.
This solution was introduced in 2017~\cite{IABAdsTxt}
and in spite of widespread adoption, the general trend of
fraud has done anything but relent.

The apparent void of solutions to this ever-growing problem has caused
many companies to turn to any technology that can promise a potential solution.
Given the nature of advertising fraud, blockchain seems a rational solution.
Billions of dollars are securely transferred every day using these
technologies in a transparent and auditable fashion~\cite{EthDVT}.
The reality is, although many solutions work at the pilot level, the
success of these solutions is dependent upon the completely unrealistic
settings in which these pilots are executed.
The core problem any AdTech blockchain solution must face is massive scale.
This scale is what lacks in the pilot environment.

Although many blockchain projects have pitched themselves as the savior
of AdTech, no solutions have, as of now, addressed the problem in a
scalable manner.
The repeated theme of many AdTech approaches to blockchain necessitates
injecting large volumes of data into a blockchain system with the
intent of bringing transparency to the supply chain.
These projects seemed to have forgotten the hard-learned lessons
surrounding blockchain scalability, or the inherent
lack thereof~\cite{BlockchainScaling}.
Another often cited mechanism of revolutionizing the AdTech space
through blockchain involves moving the real time bidding process into
smart contract systems.
The promise of such systems is to remove the middlemen of the supply
chain while making the open bid process more transparent.
Although noble in intent, these companies failed to realize that
distributed consensus takes time.
Blockchains are unable to produce blocks fast enough to meet the
sub-second demands of AdTech, not to mention the millions of requests
per second~\cite{BlockchainTPS}.
The good news is AdTech can benefit from blockchain by building systems
that address the heart of the problem.

At the core of advertising, fraud is a very old problem.
A problem that has been partially addressed many times in many
different ways.
That problem is data authenticity and integrity.
The most recent proposal of ads.cert~\cite{AdsDCert} reflects this reality.
In order to prevent fraud, the information being exchanged in the real
time bidding supply chains of modern AdTech must be protected against
manipulation and this information must have provable origination.
The IAB has unfortunately failed to see the inherent flaw in the system
they have proposed.
Similar to the fact that AdTech blockchain companies are still largely
struggling with the modern realities of the very technology they depend
upon, the specification fails to adequately
cover the ever growing subset of the advertising market that is not
directly attributable to a web domain.
Any solution that hopes to be more than a band-aid for the current
problems of AdTech must operate in a manner that will be viable for
mobile applications, over-the-top streaming services, and even mediums
we have not as of yet contemplated.
Further, these solutions are needed today.
By writing ads.cert into openRTB 3.X and not providing backward
compatibility, adoption will be significantly delayed.
