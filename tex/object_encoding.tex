\section{Canonical Object Encoding}

As with any blockchain system, a canonical encoding of objects and
messages is required to allow validation of cryptographic signatures
across systems.
Rather than focus time on solving the problem of building a canonical
encoding, we elected to select an existing robust encoding mechanism.
This allowed us to focus more time on difficult problems that are more
relevant to our direct intent.
This also allowed the system to be replicated in other programming
languages more readily by selecting a cross language encoding mechanism.

In order to achieve these goals, Cap'n Proto has been selected as the
canonical serialization format.
For those readers who may be unfamiliar with Cap'n Proto, this protocol
was built by the same engineer that designed Protobufs for Google.
The most important difference between Protobufs and Cap'n Proto, for
the MadNetwork application, is the fact that all Cap'n Proto objects
have a canonical encoding that is preserved even under additive changes
of the object definitions~\cite{CapnProto}.
This protocol has been ported to many languages and has been used in
production systems by enterprise companies such as
CloudFlare~\cite{CapnProtoNews}.
This protocol is only used for the serialization of objects and our
system does not integrate any of the RPC mechanisms associated with the
Cap'n Proto specification.

The selection of this encoding scheme is important to the following
sections.
It influenced the manner in which objects have been constructed to
reflect the operation of the encoding scheme.
This may be seen in the composition of objects to allow complex
cryptographic proofs to be formed in an elegant manner.
